\documentclass{article}

\usepackage{fullpage}

\usepackage[utf8]{inputenc}
\usepackage[T1]{fontenc}

\usepackage{amsmath,amssymb,amsthm}
  \allowdisplaybreaks % allow page breaks inside displayed equations
\usepackage[ruled,linesnumbered]{algorithm2e}
\usepackage{booktabs}
\usepackage{comment}
\usepackage{hyperref,cleveref}
\usepackage{mathtools}
\usepackage{prftree}
\usepackage{stmaryrd}
\usepackage{xspace}

\theoremstyle{definition}
  \newtheorem{definition}{Definition}
\theoremstyle{theorem}
  \newtheorem{theorem}{Theorem}

% algorithm2e
\SetKwInput{KwParam}{Parameters}
\SetKwInput{KwDescr}{Description}

\newcommand{\K}{\ensuremath{\mathbb{K}}\xspace}
\newcommand{\ps}{\ensuremath{\boldsymbol{\pi}^\textbf{2}}\xspace}
\newcommand{\imp}{\to} % implication (logic)
\newcommand{\dimp}{\leftrightarrow} % iff (logic)
\newcommand{\xto}[1]{\xrightarrow{\text{#1}}} % rightarrow with text above
\newcommand{\prule}[1]{(\textsc{#1})\xspace} % proof rule names
  \newcommand{\pruleLuk}{\prule{Łukasiewicz}}
  \newcommand{\pruleMP}{\prule{Modus Ponens}}
  \newcommand{\pruleQ}{\prule{Quantifier}}
  \newcommand{\pruleG}{\prule{Generalization}}
  \newcommand{\prulePOR}{\prule{Propagation$_{\lor}$}}
  \newcommand{\prulePORL}{\prule{Propagation$_{\lor,\textnormal{left}}$}}
  \newcommand{\prulePORR}{\prule{Propagation$_{\lor,\textnormal{right}}$}}
  \newcommand{\prulePORN}{\prule{Propagation$_{\lor,\textnormal{nested}}$}}
  \newcommand{\prulePEX}{\prule{Propagation$_{\exists}$}}
  \newcommand{\prulePEXL}{\prule{Propagation$_{\exists,\textnormal{left}}$}}
  \newcommand{\prulePEXR}{\prule{Propagation$_{\exists,\textnormal{right}}$}}
  \newcommand{\prulePEXN}{\prule{Propagation$_{\exists,\textnormal{nested}}$}}
  \newcommand{\pruleFrame}{\prule{Frame}}
  \newcommand{\pruleFrameN}{\prule{Frame$_{\textnormal{nested}}$}}
  \newcommand{\pruleFrameL}{\prule{Frame$_{\textnormal{left}}$}}
  \newcommand{\pruleFrameR}{\prule{Frame$_{\textnormal{right}}$}}
  \newcommand{\pruleSubst}{\prule{Substitution}}
  \newcommand{\pruleKT}{\prule{Knaster Tarski}}
  \newcommand{\prulePFix}{\prule{Pre-Fixpoint}}
  \newcommand{\pruleEX}{\prule{Existence}}
  \newcommand{\pruleSingleV}{\prule{Singleton Variable}}
\newcommand{\EV}{EV} % element variable set
\newcommand{\SV}{SV} % set variable set
\newcommand{\ld}{\,.\,} % used in exists x . phi and similar situations
\newcommand{\Slash}{/\!/\xspace} % //, used to begin a line comment
\newcommand{\FV}[1]{FV(#1)} % free variables
\newcommand{\mkappC}[2]{#1[#2]} % make an application context
  \newcommand{\appC}[1]{\mkappC{C}{#1}}    % C[...]
  \newcommand{\appCa}[1]{\mkappC{C_1}{#1}} % C1[...]
  \newcommand{\appCb}[1]{\mkappC{C_2}{#1}} % C2[...]
\newcommand{\hole}{\square} % the hole variable
\newcommand{\pr}[1]{\left\langle #1 \right\rangle} % <...>
\newcommand{\code}[1]{\textnormal{\texttt{#1}}} % code
  \newcommand{\cdpc}{\code{pc}} % proof checker program
  \newcommand{\cdsucc}{\code{success}} % success flag
  \newcommand{\cdfail}{\code{failure}} % failure flag
  \newcommand{\cdacc}{\code{accept}}   % accept flag
  \newcommand{\cdrej}{\code{reject}}   % reject flag
  \newcommand{\cdsum}{\code{sum.imp}\xspace} % sum program
  \newcommand{\cdassump}{\code{assump}}
  \newcommand{\cdax}{\code{ax}}
  \newcommand{\cdmp}{\code{mp}}
  \newcommand{\cdother}{\code{other}}
\newcommand{\mkth}[1]{\Gamma^{\mathsf{#1}}} % make a logical theory
  \newcommand{\GammaIMP}{\mkth{IMP}}
\newcommand{\ceil}[1]{\left\lceil #1 \right\rceil}
\newcommand{\floor}[1]{\left\lfloor #1 \right\rfloor}
% useful formulas with sub/super-scripts
\newcommand{\varphisum}{\varphi_\cdsum} % \varphi_sum
% mathcal
\newcommand{\AAc}{\mathcal{A}}
\newcommand{\CC}{\mathcal{C}}
\newcommand{\DD}{\mathcal{D}}
\newcommand{\II}{\mathcal{I}}
\newcommand{\OO}{\mathcal{O}}
\newcommand{\PP}{\mathcal{P}}
\newcommand{\VV}{\mathcal{V}}
\newcommand{\TT}{\mathcal{T}}
\newcommand{\MM}{\mathcal{M}}
\newcommand{\NN}{\mathcal{N}}
% mathbb
\newcommand{\NNN}{\mathbb{N}}
% mathbb

% ZK
\newcommand{\ZKprover}{\PP}
  \newcommand{\PPbasic}{\PP_\code{basic}}
\newcommand{\ZKverifier}{\VV}
  \newcommand{\VVbasic}{\VV_\code{basic}}
\newcommand{\pibasic}{\pi_\code{basic}}
\newcommand{\pibn}[1]{\pi_{\code{basic},#1}}
\newcommand{\MT}{\TT}
\newcommand{\MTGamma}{\TT_\Gamma}
\newcommand{\MTPi}{\TT_\Pi}
\newcommand{\cMT}{c_\MT}
\newcommand{\cPi}{c_\Pi}
\newcommand{\oPi}{o_\Pi}
\newcommand{\cGamma}{c_\Gamma}
\newcommand{\oGamma}{o_\Gamma}
% nonlogical symbols
\newcommand{\nlsymbol}[1]{\mathsf{#1}}
  \newcommand{\nlzero}{\nlsymbol{zero}}
  \newcommand{\nlsucc}{\nlsymbol{succ}}
% concrete (non-meta) variables
\newcommand{\cv}[1]{\mathsf{#1}}
  \newcommand{\cvx}{\cv{x}}
  \newcommand{\cvy}{\cv{y}}
  \newcommand{\cvX}{\cv{X}}
% basic proof checking procedures
\newcommand{\bp}[1]{\code{basic\_prove}_\code{#1}}
  \newcommand{\bpassump}{\bp{assump}}
  \newcommand{\bpax}{\bp{ax}}
  \newcommand{\bpmp}{\bp{mp}}
  \newcommand{\bpgen}{\bp{gen}}
  \newcommand{\bpframe}{\bp{frame}}
  \newcommand{\bpsubst}{\bp{subst}}
  \newcommand{\bpkt}{\bp{kt}}
  \newcommand{\bptop}{\code{basic\_prove}}
% encoding
\newcommand{\en}[1]{\left\llbracket #1 \right\rrbracket}
  \newcommand{\enGamma}{\en{\Gamma}}
  \newcommand{\enGamman}[1]{\en{\Gamma_{#1}}}
  \newcommand{\envarphi}{\en{\varphi}}
  \newcommand{\envarphin}[1]{\en{\varphi_{#1}}}
  \newcommand{\enpsi}{\en{\psi}}
  \newcommand{\enpsin}[1]{\en{\psi_{#1}}}
  \newcommand{\enDelta}{\en{\Delta}}
  \newcommand{\enDeltan}[1]{\en{\Delta_{#1}}}
% hash and commitment
\newcommand{\hof}[1]{h(#1)}
\newcommand{\Com}{\code{Com}}
\newcommand{\CSCommit}{\code{CS.Commit}}
\newcommand{\CSVerCommit}{\code{CS.VerCommit}}
\newcommand{\CSSetup}{\code{CS.Setup}}
\newcommand{\VCCommit}{\code{VC.Commit}}
\newcommand{\VCVerCommit}{\code{VC.VerCommit}}
\newcommand{\VCSetup}{\code{VC.Setup}}
\newcommand{\ck}{\mathit{ck}}
  \newcommand{\ckGamma}{\ck_\Gamma}
  \newcommand{\ckPi}{\ck_\Pi}
\newcommand{\Ispace}{\II}
\newcommand{\Ospace}{\OO}
\newcommand{\Cspace}{\CC}
\newcommand{\Adversary}{\AAc}
\newcommand{\negl}{\code{negl}}
% data space
\newcommand{\Data}{\DD}
\newcommand{\Dpattern}{\DD_\textit{pattern}}
\newcommand{\Dtheory}{\DD_\textit{theory}}
% proof network
\newcommand{\PNet}{\NN}
\newcommand{\Ninit}{\mathit{IN}}
\newcommand{\Naggr}{\mathit{AN}}
\newcommand{\Npr}{\mathit{PN}}
  \newcommand{\Nprn}[1]{\Npr_{#1}}
% vectors
\newcommand{\vecu}{\vec{u}}

\title{Matching Logic Proofs Meet Succinct Cryptographic Proofs}
\author{}
\date{}

\begin{document}

\maketitle

\begin{abstract}
We first present the syntax and proof system of matching logic.
Then, we introduce the research problem of producing embarrassingly parallel 
cryptographic proofs for matching logic. 
\end{abstract}

\section{Matching Logic}

We present matching logic syntax in \Cref{sec:ml_syntax}
and its proof system in \Cref{sec:ml_proof_system}. 
We introduce the matching logic proof checker in \Cref{sec:ml_proof_checker}. 

\subsection{Matching Logic Syntax}
\label{sec:ml_syntax}

We fix countably infinite sets $\EV$ and $\SV$, where
\begin{enumerate}
\item $\EV$ is a set of \emph{element variables},
      often denoted $x$, $y$, and $z$. 
\item $\SV$ is a set of \emph{set variables},
      often denoted $X$, $Y$, and $Z$.
\end{enumerate}
We assume that $\EV$ and $\SV$ are disjoint, i.e., $\EV \cap \SV = \emptyset$. 
A \emph{variable}, often denoted $u$ and $v$,
is either an element variable or a set variable.
We write $u \equiv v$ iff $u$ and $v$ denote the same variable.
We write $u \not\equiv v$ iff $u$ and $v$ denote two distinct variables. 

\begin{definition}
\label{def:ml_patterns}
A \emph{signature} $\Sigma$ is a set whose elements are called 
\emph{symbols}, often denoted $\sigma$. 
The set of \emph{$\Sigma$-patterns},
or simply \emph{patterns} when $\Sigma$ is understood,
is inductively defined by the following grammar:
\begin{align*}
\varphi 
&\Coloneqq x 
  && \text{\Slash Element Variable} 
\\&\ \ \ \mid X 
  && \text{\Slash Set Variable}
\\&\ \ \ \mid \sigma 
  && \text{\Slash Symbol}
\\&\ \ \ \mid \varphi_1 \, \varphi_2
  && \text{\Slash Application}
\\&\ \ \ \mid \varphi_1 \imp \varphi_2
  && \text{\Slash Implication}
\\&\ \ \ \mid \exists x \ld \varphi
  && \text{\Slash Existential Quantification}
\\&\ \ \ \mid \mu X \ld \varphi
  && \text{\Slash Least Fixpoint}
\end{align*}
where $\mu X \ld \varphi$ requires that $\varphi$ is positive in $X$,
which means that $X$ is not nested in an odd number 
of times on the left-hand side of an implication. 
\end{definition}

The following are some well-formed least fixpoint patterns:
\begin{enumerate}
\item $\mu X \ld X$
\item $\mu X \mu Y \ld X$
\item $\mu X \ld (Y \imp \bot)$
\item $\mu X \ld ((X \imp \bot) \imp \bot)$
\item $\mu X \ld \exists y \ld ((X \imp y) \imp (y \imp X))$
\item $\mu X \ld (\sigma \, (X \imp \bot)) \imp \bot$
\end{enumerate}
The following are some ill-formed least fixpoint patterns:
\begin{enumerate}
\item $\mu X \ld (X \imp \bot)$
\item $\mu X \ld (X \imp X)$
\item $\mu X \ld (\sigma \, X) \imp \bot$
\end{enumerate}

Application is left-associative;
for example, $\varphi_1 \, \varphi_2 \, \varphi_3$ means
$(\varphi_1 \, \varphi_2) \, \varphi_3$.
Implication is right-associative;
for example, $\varphi_1 \imp \varphi_2 \imp \varphi_3$ means
$\varphi_1 \imp (\varphi_2 \imp \varphi_3)$. 
Application has a higher precedence than implication;
for example, $\varphi_1 \imp \varphi_2 \, \varphi_3$ means
$\varphi_1 \imp (\varphi_2 \, \varphi_3)$. 
The scope of $\exists x$ and $\mu X$ goes as far to right as possible,
for example, $\exists x \ld \varphi_1 \imp \varphi_2$ means
$\exists x \ld (\varphi_1 \imp \varphi_2)$,
and not $(\exists x \ld \varphi_1) \imp \varphi_2$.

\Cref{def:ml_patterns} includes 7 \emph{primitive constructs} of matching logic.
Besides these primitive constructs, we also allow \emph{notations}
(also called 
abbreviations, short-hands, derived constructs, or syntactic sugar),
which are constructs that can be defined using the primitive ones. 
The following are some commonly used notations:
\begin{align}
\top 
&\equiv \exists x \ld x
&&\text{\Slash Logical True}
\\
\bot 
&\equiv \mu X \ld X
&&\text{\Slash Logical False}
\\
\neg \varphi 
&\equiv \varphi \imp \bot
&&\text{\Slash Negation}
\\
\varphi_1 \lor \varphi_2
&\equiv (\neg \varphi_1) \imp \varphi_2
&&\text{\Slash Disjunction}
\\
\varphi_1 \land \varphi_2
&\equiv \neg(\neg \varphi_1 \lor \neg \varphi_2)
&&\text{\Slash Conjunction}
\\
\varphi_1 \dimp \varphi_2
&\equiv (\varphi_1 \imp \varphi_2) \land (\varphi_2 \imp \varphi_1)
&&\text{\Slash ``If-and-Only-If''}
\\
\forall x \ld \varphi
&\equiv \neg \exists x \ld \neg \varphi
&&\text{\Slash Universal Quantification}
\\
\nu X \ld \varphi
&\equiv \neg \mu X \ld \neg \varphi[\neg X / X]
&&\text{\Slash Greatest Fixpoint}
\end{align}
where $\varphi[\neg X / X]$ is the pattern obtained by
substituting $\neg X$ for $X$ in $\varphi$. 
We formally define substitution in \Cref{def:substitutions}. 



Among the primitive constructs, $\exists x$ and $\mu X$ are \emph{binders}. 
We define the concepts that are important to binders:
\emph{free variables}, \emph{$\alpha$-renaming}, and
\emph{capture-avoiding substitution}.
These definitions are standard
but we still present all the technical details to make this article
self-contained. 
Readers who are already familiar with these concepts can skip
the rest and jump to \Cref{sec:ml_proof_system}. 

\begin{definition}
Given a pattern $\varphi$,
we inductively
define its set of \emph{free variables}, denoted $\FV{\varphi}$, as follows:
\begin{enumerate}
\item $\FV{x} = \{x\}$
\item $\FV{X} = \{X\}$
\item $\FV{\sigma} = \emptyset$
\item $\FV{\varphi_1 \, \varphi_2} = \FV{\varphi_1} \cup \FV{\varphi_2}$
\item $\FV{\varphi_1 \imp \varphi_2} = \FV{\varphi_1} \cup \FV{\varphi_2}$
\item $\FV{\exists x \ld \varphi} = \FV{\varphi} \setminus \{x\}$
\item $\FV{\mu X \ld \varphi} = \FV{\varphi} \setminus \{X\}$
\end{enumerate}
We say that $\varphi$ is \emph{closed} if $\FV{\varphi} = \emptyset$. 
\end{definition}

We can categorize the occurrences of a variable in a pattern
into \emph{free} and \emph{bound} occurrences,
based on whether the occurrence is within the scope
of a corresponding binder. 

\begin{definition}
For any $x$, $X$, and $\varphi$,
\begin{enumerate}
\item A \emph{free occurrence} of $x$ in $\varphi$ 
      is an occurrence that is not within the scope of any $\exists x$ binder;
\item A \emph{free occurrence} of $X$ in $\varphi$
      is an occurrence that is not within the scope of any $\mu X$ binder;
\item A \emph{bound occurrence} of $x$ is an occurrence
      within the scope of an $\exists x$ binder;
\item A \emph{bound occurrence} of $X$ is an occurrence
      within the scope of a $\mu X$ binder.
\end{enumerate}
We say that a variable is \emph{bound} in a pattern if 
all occurrences are bound. 
We say that a variable is \emph{fresh} w.r.t. a pattern if
there are no occurrences, free or bound, of the variable. 
\end{definition}

As an example, consider the pattern
$x \land \exists x \exists y \ld x \land y$
where $x \not\equiv y$. 
This pattern 
has two occurrences of $x$ and one occurrence of $y$.\footnote{Note that we do not count $\exists x$ and $\exists y$ as occurrences, although some works do count them and call them \emph{binding} occurrences. In this article, we only consider free and bound occurrences and regard
$\exists x$ and $\mu X$ as syntactic constructs.}
The left-most $x$ is a free occurrence.
The right-most $x$ and the right-most $y$ are both bound occurrences. 

We can rename a bound variable to a fresh variable. 
Such operation is called \emph{$\alpha$-renaming}. 
For example, the above example pattern 
$x \land \exists x \exists y \ld x \land y$
can be $\alpha$-renamed into
$x \land \exists z \exists y \ld z \land y$, 
where $z \not\equiv x$ and $z \not\equiv y$. 

\begin{definition}
Let $\varphi_1$ and $\varphi_2$ be patterns. 
We say that $\varphi_1$ and $\varphi_2$
are \emph{$\alpha$-equivalent}, written $\varphi_1 \equiv_\alpha \varphi_2$,
iff $\varphi_1$ can be transformed into $\varphi_2$ after applying
a finite number of $\alpha$-renamings. 
\end{definition}

It is straightforward to show that $\alpha$-equivalence
is indeed an equivalence relation. 
In fact, $\alpha$-equivalence is the reflexive, commutative, and transitive
closure of $\alpha$-renaming. 

Sometimes, it takes more than one $\alpha$-renaming to transform
a pattern into an $\alpha$-equivalent one. 
For example, $x \land \exists x \exists y \ld x \land y$
and $x \land \exists y \exists x \ld y \land x$ are $\alpha$-equivalent,
but it takes three $\alpha$-renamings to transform the former into the latter:
\begin{align}
x \land \exists x \exists y \ld x \land y
\xto{$\alpha$-renaming}
x \land \exists z \exists y \ld z \land y
\xto{$\alpha$-renaming}
x \land \exists z \exists x \ld z \land x
\xto{$\alpha$-renaming}
x \land \exists y \exists x \ld y \land x
\end{align}
In the above $z$ is fresh; that is, $z \not\equiv x$ and $z \not\equiv y$. 

Note that a notation can be a binder. 
For example, $\forall x$, which is a notation defined by
$\forall x \ld \varphi \equiv \neg \exists x \ld \neg \varphi$,
is a binder.
Therefore, all occurrences of $x$ in $\forall x \ld \varphi$ are bound,
and $\FV{\forall x \ld \varphi} = \FV{\varphi} \setminus \{x\}$. 

Next, we define substitution. 
Because we have binders, we need to be careful to avoid the
infamous phenomenon of free variable capture,
illustrated by the following example.
Suppose $\varphi \equiv \exists x \ld x \imp y$
where $x \not\equiv y$
and we want to substitute $y$ for $x$ in $\varphi$.
If we do it blindly by replacing every (free) occurrence of $y$ with $x$,
we get
$\exists x \ld x \imp x$. 
This is unintended because $y$, which is previously not within the scope of $\exists x$, is now bound. 
The correct way to do substitution 
is to first $\alpha$-rename
$\exists x \ld x \imp y$ into $\exists z \ld z \imp y$, for some fresh variable $z$. 
Then, we apply the substitution and obtain the correct result
$\exists z \ld z \imp x$. 
Note that this time, $x$ is no longer bound, which is intended. 

The operation described above is called \emph{capture-avoiding substitution}. 
As we can see from the above example, capture-avoiding substitution needs
$\alpha$-renaming to avoid free variable capture.
In the literature, there are two standard
approaches to define capture-avoiding substitution:
\begin{enumerate}
\item We can define capture-avoiding substitution as a partial function 
      that is undefined in the case of free variable capture.
      In this approach, $\alpha$-renaming happens explicitly to avoid free variable capture. 
\item We can define capture-avoiding substitution as a total function
      on $\alpha$-equivalence classes. In this approach, $\alpha$-renaming
      happens implicitly to avoid free variable capture. 
\end{enumerate}
In this article, we take the second approach;
that is, we always work with $\alpha$-equivalence classes 
throughout the article. 


\begin{definition}
\label{def:substitutions}
Let $\varphi$ and $\psi$ be patterns and $v$ and $u$ be variables.
We define $\varphi[\psi/v]$ to be the result of substituting
$\psi$ for $v$ in $\varphi$ as follows:
\begin{enumerate}
\item $v[\psi/v] = \psi$
\item $u[\psi/v] = u$ if $u \not\equiv v$
\item $\sigma[\psi/v] = \sigma$
\item $(\varphi_1 \, \varphi_2)[\psi/v] 
       = (\varphi_1[\psi/v]) \, (\varphi_2[\psi/v])$
\item $(\varphi_1 \imp \varphi_2)[\psi/v]
       =\varphi[\psi/v] \imp \varphi_2[\psi/v]$
\item $(\exists x \ld \varphi)[\psi/x] = \exists x \ld \varphi$
\item $(\exists x \ld \varphi)[\psi/v] = \exists y \ld \varphi[y/x][\psi/v]$,
      if $v \not\equiv x$ and $y$ is fresh
\item $(\mu X \ld \varphi)[\psi/X] = \mu X \ld \varphi$
\item $(\mu X \ld \varphi)[\psi/v] = \mu Y \ld \varphi[Y/X][\psi/v]$,
      if $v \not\equiv X$ and $Y$ is fresh
\end{enumerate}
\end{definition}

Before we move on, we have a remark on the presentation of binders. 
Binders and their binding behaviors are known to be a non-trivial matter
in the study of formal logic and its syntax. 
Our presentation in this article 
is a classic textbook presentation 
where variables are represented by their names. 
A \emph{nameless} presentation such as De Bruijn index 
(\url{https://en.wikipedia.org/wiki/De_Bruijn_index})
is an encoding method that represents bound variables by numbers instead of their names. 
These numbers, called De Bruijn indices, denote the number of (nest) binders
on top of a bound variable. 
De Bruijn index has the advantage that $\alpha$-equivalent terms
have the same encoding
so checking $\alpha$-equivalence is the same as checking syntactic equality. 
Locally nameless approach
(\url{https://chargueraud.org/research/2009/ln/main.pdf})
uses De Bruijn indices only for bound variables
and retains names for free variables for readability. 
Nominal approach (\url{https://www.sciencedirect.com/science/article/pii/S089054010300138X})
uses \emph{swapping} and \emph{freshness} as the primitives notions/operations
regarding binders, instead of using 
$\alpha$-equivalence and capture-avoiding substitution as the primitive. 
The reason is that swapping and freshness have simpler definitions
than $\alpha$-equivalence and (especially) capture-avoiding substitution.
Higher-order abstract syntax (HOAS) studies data structures
that explicit expose the relationship between variables and their corresponding binders. 
There is extensive study on matching and unification algorithms
modulo $\alpha$-equivalence in the literature on
nominal and HOAS approaches. 

We are open to all these approaches regarding binders. 

\subsection{Matching Logic Proof System}
\label{sec:ml_proof_system}

We present the entire matching logic proof system in 
\Cref{fig:ml-proof-system}.
To understand the proof system, we first need the following definition.

\begin{definition}
Let $\hole$ be a distinguished element variable.
An \emph{application context} is a pattern
where $\hole$ has exactly one occurrence, which is under 
a number of nested applications. 
More formally, 
\begin{enumerate}
\item $\hole$ is an application context, called the \emph{identity context};
\item if $C$ is an application context and $\varphi$ is a pattern
      with no occurrences of $\hole$,
      both $(C \, \varphi)$ and $(\varphi \, C)$ are application contexts. 
\end{enumerate}
For an application context $C$ and a pattern $\psi$,
we write $C[\psi]$ to mean $C[\psi/\hole]$. 
\end{definition}

\begin{figure}[t]
\centering
{\renewcommand{\arraystretch}{2}
\begin{tabular}{rcl}
\toprule
\pruleLuk
&
$((\varphi_1 \imp \varphi_2) \imp \varphi_3) 
   \imp ((\varphi_3 \imp \varphi_1) \imp (\varphi_4 \imp \varphi_1))$
& axiom schema
\\
\pruleMP
&
$\prftree{\varphi_1 \imp \varphi_2}{\varphi_1}{\varphi_2}$
& rule of inference with 2 premises
\\
\pruleQ
&
$\varphi[y/x] \imp \exists x \ld \varphi$
& axiom schema
\\
\pruleG
&
$\prftree[r]{\quad if $x \not\in \FV{\varphi_2}$}
  {\varphi_1 \imp \varphi_2}
  {(\exists x \ld \varphi_1) \imp \varphi_2}$
& rule of inference with 1 premise
\\
\prulePOR
&
$\appC{\varphi_1 \lor \varphi_2} \imp \appC{\varphi_1} \lor \appC{\varphi_2}$
& axiom schema
\\
\prulePEX
&
$\appC{\exists x \ld \varphi} \imp \exists x \ld \appC{\varphi}$
\quad if $x \not\in \FV{\appC{\exists x \ld \varphi}}$
& axiom schema
\\
\pruleFrame
&
$\prftree{\varphi_1 \imp \varphi_2}{\appC{\varphi_1} \imp \appC{\varphi_2}}$
& rule of inference with 1 premise
\\
\pruleSubst
&
$\prftree{\varphi}{\varphi[\psi/X]}$
& rule of inference with 1 premise
\\
\prulePFix
&
$\varphi[\mu X \ld \varphi / X] \imp \mu X \ld \varphi$
& axiom schema
\\
\pruleKT
&
$\prftree{\varphi[\psi/X] \imp \psi}{(\mu X \ld \varphi) \imp \psi}$
& rule of inference with 1 premise
\\
\pruleEX
&
$\exists x \ld x$
& axiom schema
\\
\pruleSingleV
&
$\neg (\appCa{x \land \varphi} \land \appCb{x \land \neg\varphi})$
& axiom schema
\\\bottomrule
\end{tabular}
} % close \renewcommand{\arraystretch}{2}
\caption{Matching Logic Proof System}
\label{fig:ml-proof-system}
\end{figure}

An \emph{axiom schema} (or \emph{schema}) is 
a pattern with the occurrence of \emph{metavariables}, such as $x$, $X$, $C$, and $\varphi$.
For example, 
\pruleLuk
is a schema with four metavariables
$\varphi_1$, $\varphi_2$, $\varphi_3$, and $\varphi_4$.
\[
\pruleLuk
\quad
((\varphi_1 \imp \varphi_2) \imp \varphi_3) 
   \imp ((\varphi_3 \imp \varphi_1) \imp (\varphi_4 \imp \varphi_1))
\]
An \emph{instance} of a schema is the pattern obtained by
instantiating all the metavariables in the said schema. 
The following are all instances of \pruleLuk.
\begin{align*}
((\bot \imp \top) \imp \nlzero) 
   \imp ((\nlzero \imp \bot) \imp ((\nlsucc \, \nlzero) \imp \bot))
\\
((\bot \imp \top) \imp (\nlzero \imp \bot)) 
   \imp (((\nlzero \imp \bot) \imp \bot) \imp (\bot \imp \bot))
\end{align*}
where $\nlzero,\nlsucc \in \Sigma$ are two (concrete) symbols in the signature. 
\pruleQ is a schema with metavariables $x$, $y$, and $\varphi$. 
\[
\pruleQ \quad \varphi[y/x] \imp \exists x \ld \varphi
\]
The following are all instances of \pruleQ.
\begin{align*}
(\cvy \land \cvy) \imp \exists \cvx \ld (\cvx \land \cvy)
\\
(\cvx \land \cvy) \imp \exists \cvx \ld (\cvx \land \cvy)
\end{align*}
where $\cvx$ and $\cvy$ are concrete (non-meta) element variables. 

A \emph{rule of inference (schema)} is a pair of
some \emph{premises} and one \emph{conclusion},
all possibly sharing some metavariables.
Rules of inference are written as shown below:
\[
\prftree{\varphi_1}{\cdots}{\varphi_n}{\psi}
\]
We write the premises above the line and the conclusion below the line. 
An \emph{instance} of a rule of inference is obtained by
instantiating all the metavariables in the rule of inference. 

The matching logic proof system as shown in \Cref{fig:ml-proof-system}
has 7 axiom schemas and 5 rules of inference. 
We call axiom schemas and rules of inference \emph{proof rules}.
Matching logic has 7 + 5 = 12 proof rules. 

\begin{definition}
A \emph{theory} is a pair $(\Sigma,\Gamma)$
where $\Sigma$ is a signature and $\Gamma$ is a set of $\Sigma$-patterns, 
called \emph{non-logical axioms}. 
When $\Sigma$ can be understood from the context,
we abbreviate $(\Sigma, \Gamma)$ as $\Gamma$. 
\end{definition}

\begin{definition}
\label{def:Hilbert-proof}
Let $\Gamma$ be a theory.
A \emph{Hilbert $\Gamma$-proof (for $\varphi_n$)} is a finite sequence of patterns:
\begin{equation}
\pr{\varphi_1 , \varphi_2 , \dots , \varphi_n}
\end{equation}
where $n \ge 1$, and for every $1 \le i \le n$, one of the following holds:
\begin{enumerate}
\item $\varphi_i \in \Gamma$;
\item $\varphi_i$ is an instance of one of the 7 axiom schemas;
\item $\varphi_i$ can be derived using one of the 5 rules of inference
      from previous derived patterns 
      (i.e., those in $\{\varphi_1,\dots,\varphi_{i-1}\}$) as premises.  
\end{enumerate}
We write $\Gamma \vdash \varphi$, meaning that $\varphi$ is provable/derivable from $\Gamma$, 
iff there exists a Hilbert $\Gamma$-proof for $\varphi$. 
\end{definition}

Note that a Hilbert $\Gamma$-proof for $\varphi$ can always be re-arranged
such that it starts with the axioms in $\Gamma$, followed by logical axioms
and conclusions of applying the rules of inference. 
More formally, if $\varphi$ has a Hilbert $\Gamma$-proof, i.e., $\Gamma \vdash \varphi$, then there must exist
a Hilbert $\Gamma$-proof
\begin{equation}
\pr{\underbrace{\varphi_1 , \varphi_2 , \dots, \varphi_k}_\text{axioms in $\Gamma$},
\underbrace{\varphi_{k+1},\dots,\varphi_{l}}_\text{logical axioms},
\underbrace{\varphi_{l+1}, \dots , \varphi_n}_\text{conclusions}}
\end{equation}
such that $\varphi_1,\dots,\varphi_k \in \Gamma$,
$\varphi_{k+1},\dots,\varphi_l$ are logical axioms,
$\varphi_{l+1},\dots,\varphi_n$ are conclusions of the rules of inferences,
and $\varphi_n \equiv \varphi$. 
In this proof, $\varphi_1,\dots,\varphi_k$ are from $\Gamma$ so they are public.
The last pattern $\varphi_n$ is the theorem being proved so it is also public. 
The intermediate patterns $\varphi_{k+1},\dots,\varphi_{n-1}$ 
are private or hidden. 

As a concluding remark, we mention that the precise presentation of
the proof system can be (slightly) modified, without changing the provability relation $\Gamma \vdash \varphi$. 
For example, 
we can replace the \pruleLuk axiom schema
\[
\pruleLuk
\quad
((\varphi_1 \imp \varphi_2) \imp \varphi_3) 
   \imp ((\varphi_3 \imp \varphi_1) \imp (\varphi_4 \imp \varphi_1))
\]
with a single axiom (i.e., not a schema):
\[
\prule{Łukasiewicz'}
\quad
((\cvX_1 \imp \cvX_2) \imp \cvX_3) 
   \imp ((\cvX_3 \imp \cvX_1) \imp (\cvX_4 \imp \cvX_1))
\]
where $\cvX_1,\dots,\cvX_4$ are four (different) concrete set variables. 
Any instance of \pruleLuk is derivable from \prule{Łukasiewicz'}
and the \pruleSubst rule of inference. 
As another example, we can replace
\pruleFrame with two rules of inference
\begin{align*}
\pruleFrameL\quad
  & \prftree{\varphi_1 \imp \varphi_2}
            {(\varphi_1 \, \psi) \imp (\varphi_2 \, \psi)}
\\
\pruleFrameR\quad
  & \prftree{\varphi_1 \imp \varphi_2}
            {(\psi \, \varphi_1) \imp (\psi \, \varphi_2)}
\end{align*}
While now we have two rules instead of one, 
we avoid the use of application contexts (i.e., $C[\dots]$),
which will make proof checking easier. 
However, we will get longer proofs, because every \pruleFrame step
will need a number of \pruleFrameL/\pruleFrameL steps,
depending on the number of symbols in $C$. 

To conclude, there is some flexibility in the design of the matching logic proof system. 
The main criteria here is the simplicity of the generation and verification
of the succinct cryptographic proofs of matching logic proofs,
as discussed in \Cref{sec:crypto_proofs}. 

\subsection{Matching Logic Proof Checker}
\label{sec:ml_proof_checker}

Generally speaking, 
a matching logic proof checker is a deterministic and terminating program
\begin{equation}
\cdpc(\Gamma, \varphi, \Pi) \in \{\cdsucc, \cdfail\}
\end{equation}
The three inputs are:
\begin{enumerate}
\item a theory $\Gamma$, which can be the axiomatization of a mathematical domain or the formal semantics of a programming or specification language; 
\item a pattern/claim $\varphi$;
\item a proof object $\Pi$, 
      which encodes a Hilbert $\Gamma$-proof for $\varphi$.
\end{enumerate}
A matching logic proof checker should satisfy the following
soundness and completeness property:\footnote{Do not confuse it 
with the soundness and completeness of a formal logic, which, intuitively speaking, 
is the equivalence between what is semantically true and what is provable. In this article, we do not care about semantics at all. The soundness and completeness property is a property about the proof checker.}
\begin{equation}
\label{eq:pc_sound_complete}
\Gamma \vdash \varphi
\quad\text{iff}\quad
\text{there exists $\Pi$ such that $\cdpc(\Gamma, \varphi, \Pi) = \cdsucc$}
\end{equation}

To implement a matching logic proof checker,
one needs to consider many factors. 
For example, one should decide an encoding/decoding mechanism for 
representing $\Gamma$, $\varphi$, and $\Pi$.
Sometimes, the proof object $\Pi$ can include additional
proof annotations to simplify the main proof checking algorithm. 
One should also formalize the metalevel operations/definitions
in the proof checking algorithms, including but not limited to
free variables, capture-avoiding substitution, and application contexts.
These metalevel operations/definitions are necessary to formulate the
matching logic proof system in \Cref{sec:ml_proof_system}. 

We have implemented a matching logic proof checker in Metamath
(\url{http://metamath.org})
in 200 lines of code.\footnote{See \url{https://github.com/runtimeverification/proof-generation/blob/main/theory/matching-logic-200-loc.mm}.}.  
Modulo all the technical and implementation details, 
the essence of the Metamath implementation is a reduction 
from checking matching logic proofs into three basic operations
over strings:
\begin{enumerate}
\item checking string equivalence;
\item string substitution; 
\item dictionary lookups.
\end{enumerate}


\section{Problem Statement}
\label{sec:crypto_proofs}

Mathematical proof objects can be very large.
For example, the proof object for
$\GammaIMP \vdash \varphisum$,
where
\begin{align}
\GammaIMP &\quad=\quad 
\text{formal semantics of IMP---a simple imperative language; 40 lines of \K}
\\
\varphisum &\quad=\quad 
\text{a claim stating that the output of the following \cdsum program is 5050}
\\
\cdsum &\quad=\quad 
\code{int n = 100, s = 0; while(-{}-n)\{s = s + n;\}; output(s);}
\end{align}
consumes 83.4 MB of storage.\footnote{See \url{https://github.com/runtimeverification/proof-generation/blob/main/examples/sum-imp-complete-proof.mm}.}
Therefore, mathematical proof objects are not good candidates for
serving as \emph{correctness certificates}.
They are too large to exchange. 
On the other hand, 
it is unnecessary to use a complete mathematical proof object
if all that we want to show is $\Gamma \vdash \varphi$.
According to \Cref{eq:pc_sound_complete},
we only need show the \emph{existence} of a mathematical proof object
that will pass the proof checker. 

The \emph{research problem} is to 
design a succinct cryptographic proof system
$(\ZKprover,\ZKverifier)$
with a prover $\ZKprover$ and a verifier $\ZKverifier$.
The prover $\ZKprover$ takes three inputs\footnote{The prover/verifier can take more inputs such as a common reference string.}, i.e., 
\begin{equation}
\ZKprover(\Gamma, \varphi, \Pi) = \pi
\end{equation}
and produces a succinct cryptographic proof $\pi$. 
The verifier $\ZKverifier$ takes $\Gamma$, $\varphi$, and $\pi$,
and decides whether it accepts or rejects the inputs, i.e.,
\begin{equation}
\ZKverifier(\Gamma, \varphi, \pi) \in \{\cdacc, \cdrej\}
\end{equation}
Such a cryptographic proof system should satisfy the following soundness and completeness property:
\begin{enumerate}
\item If $\cdpc(\Gamma,\varphi,\Pi) = \cdsucc$, we have $\ZKverifier(\Gamma,\varphi,\ZKprover(\Gamma, \varphi, \Pi)) = \cdacc$
\item If $\cdpc(\Gamma,\varphi,\Pi) = \cdfail$, we have
$\ZKverifier(\Gamma, \varphi, \ZKprover'(\Gamma, \varphi, \Pi)) = \cdrej$
for any $\ZKprover'$. 
\end{enumerate}
The above properties can be relaxed to allow exceptions with small probability
and/or require limited computation power for $\ZKprover'$. 

\section{Our Approach}

What distinguishes our task from developing a SNARK/STARK for a generic program
is the following crucial fact:
\emph{checking mathematical proof objects is an embarrassingly parallel problem.} 
Consider an arbitrary Hilbert proof
$\pr{\varphi_1,\varphi_2,\dots,\varphi_n}$. 
By \Cref{def:Hilbert-proof}, for every $1 \le i \le n$,
the pattern/claim $\varphi_i$
is either an axiom or derivable using one of the 5 non-nullary proof rules. 
If $\varphi_i$ is an axiom, checking it is \emph{local}.
If $\varphi_i$ is derivable, checking it requires to look up at most 2 existing claims.\footnote{\pruleMP has two premises. The other rules of inferece have one premise each. Logical axiom schemas have no premises. } 

We want to utilize the fact that
checking mathematical proof objects is an embarrassingly parallel problem
to optimize our succinct cryptographic proof system.
For example, we develop a circuit for every case in \Cref{def:Hilbert-proof}.
This results in 13 circuits; 12 of them are for the 12 
(non-nullary and nullary) proof rules in \Cref{fig:ml-proof-system} 
and the last one is for checking membership $\varphi \in \Gamma$,
which is Case (1) in \Cref{def:Hilbert-proof}. 
Once each proof step is checked, we add its hash to a Merkle tree. 
If all the patterns/claims are added to the tree, 
we conclude that the entire proof is correct.

\subsection{Overview}

Our approach consists of three phases. 
In the first phase, we generate a basic certificate for each Hilbert-style proof step. 
In the second phase, we ensure that the basic certificates
generated in the first phase belong to the same Hilbert-style proof,
by using Merkle trees. 
In the third phase, we aggregate the basic certificates
using SnarkPack, which is an efficient way to aggregate
any SNARK certificates in logarithmic time.  

Throughout this section, let us assume
the matching logic theorem being proved is
\begin{equation}
\Gamma \vdash \varphi_n
\end{equation}
and a corresponding Hilbert-style proof is
\begin{equation}
\Pi = \pr{\varphi_1,\dots,\varphi_n}
\end{equation}

\subsection{Phase 1: Basic Certificates}

We first introduce \emph{basic proof checking procedures} or simply
\emph{basic procedures}.
These procedures check the correctness of each individual
Hilbert-style proof steps.

We use $\en{\_}$ to denote an injective function,
called an \emph{encoding function},
from matching logic patterns and theories into data. 
For example, 
$\enGamma$ is an encoding of $\Gamma$ and $\envarphi$ is an encoding of $\varphi$. 
This way, we have the flexibility to use any encoding mechanism
(textbook encoding, de Bruijn encoding, locally nameless encoding, nominal encoding, \ldots)
and any representation method (S-expression, Metamath/RPN, a binary representation, \ldots.)
We use $\alpha$ to denote \emph{proof annotation}, which includes information
that is needed and/or useful for checking basic proof steps,
such as a substitution of meta-variables. 

We formalize the basic proof checking procedures that check the correctness of one-step Hilbert-style proofs.
We split the arguments into \emph{public arguments} (listed before ``$;$'')
and \emph{private arguments} (listed after ``$;$''). 
Note that proof annotations (denoted $\alpha$) are private because they
only simplify the proof checking procedures.
\begin{enumerate}
\item $\bpassump(\enGamma , \envarphi ; \alpha) \in \{\cdsucc,\cdfail\}$
\begin{enumerate}
\item Returns $\cdsucc$ for some $\alpha$ iff $\varphi \in \Gamma$.
\item Returns $\cdfail$ for any $\alpha$ iff otherwise. 
\end{enumerate}
\item $\bpax(\envarphi ; \alpha) \in \{\cdsucc,\cdfail\}$
\begin{enumerate}
\item Returns $\cdsucc$ for some $\alpha$ iff $\varphi$ is a logical axiom, defined in \Cref{fig:ml-proof-system}.
\item Returns $\cdfail$ for any $\alpha$ iff otherwise. 
\end{enumerate}
\item $\bpmp(\enpsin{1}, \enpsin{2}, \envarphi ; \alpha) \in \{\cdsucc,\cdfail\}$
\begin{enumerate}
\item Returns $\cdsucc$ for some $\alpha$ iff
      $\varphi$ can be proved from $\psi_1$ and $\psi_2$ using \pruleMP.\footnote{That is, $\psi_1$ is identical to $\psi_2 \imp \varphi$. One can deduce similar criteria for the other proof rules according to \Cref{fig:ml-proof-system}.}
\item Returns $\cdfail$ for any $\alpha$ iff otherwise. 
\end{enumerate}
\item $\bpgen(\enpsi, \envarphi ; \alpha) \in \{\cdsucc,\cdfail\}$
\begin{enumerate}
\item Returns $\cdsucc$ for some $\alpha$ iff
      $\varphi$ can be proved from $\psi$ using \pruleG. 
\item Returns $\cdfail$ for any $\alpha$ iff otherwise.  
\end{enumerate}
\item $\bpframe(\enpsi, \envarphi ; \alpha) \in \{\cdsucc,\cdfail\}$
\begin{enumerate}
\item Returns $\cdsucc$ for some $\alpha$ iff
      $\varphi$ can be proved from $\psi$ using \pruleFrame. 
\item Returns $\cdfail$ for any $\alpha$ iff otherwise.  
\end{enumerate}
\item $\bpsubst(\enpsi, \envarphi ; \alpha) \in \{\cdsucc,\cdfail\}$
\begin{enumerate}
\item Returns $\cdsucc$ for some $\alpha$ iff
      $\varphi$ can be proved from $\psi$ using \pruleSubst. 
\item Returns $\cdfail$ for any $\alpha$ iff otherwise.  
\end{enumerate}
\item $\bpkt(\enpsi, \envarphi ; \alpha) \in \{\cdsucc,\cdfail\}$
\begin{enumerate}
\item Returns $\cdsucc$ for some $\alpha$ iff
      $\varphi$ can be proved from $\psi$ using \pruleKT. 
\item Returns $\cdfail$ for any $\alpha$ iff otherwise.  
\end{enumerate}
\end{enumerate}

For simplicity, we can merge the above basic procedures into 
a \emph{top procedure}:
\begin{equation}
\bptop(\enGamma, \enDelta, \envarphi ; \alpha) \in \{\cdsucc,\cdfail\}
\end{equation}
such that
$\bptop(\enGamma, \enDelta, \envarphi ; \alpha)$ returns $\cdsucc$ iff
one of the following conditions holds for some $\alpha'$:
\begin{enumerate}
\item $\bpassump(\enGamma, \envarphi ; \alpha') = \cdsucc$
      and $\Delta = \emptyset$.
\item $\bpax(\envarphi; \alpha') = \cdsucc$
      and $\Delta = \emptyset$.
\item $\bpmp(\enpsin{1}, \enpsin{2}, \envarphi ; \alpha') = \cdsucc$
      and $\Delta = \{\psi_1,\psi_2\}$.
\item $\bpgen(\enpsi, \envarphi ; \alpha') = \cdsucc$
      and $\Delta = \{\psi\}$.
\item $\bpframe(\enpsi, \envarphi ; \alpha') = \cdsucc$
      and $\Delta = \{\psi\}$.
\item $\bpsubst(\enpsi, \envarphi ; \alpha') = \cdsucc$
      and $\Delta = \{\psi\}$.
\item $\bpkt(\enpsi, \envarphi ; \alpha') = \cdsucc$
      and $\Delta = \{\psi\}$.
\item $\bpkt(\enpsi, \envarphi ; \alpha') = \cdsucc$
      and $\Delta = \{\psi\}$.
\end{enumerate}
In other words, $\Delta$ is the set of premises used by a proof step. 

The following theorem states that it is sound and complete
to decompose any Hilbert-style proof into its basic proof steps. 

\begin{theorem}
Proof sequence $\pr{\varphi_1,\dots,\varphi_n}$ is a Hilbert $\Gamma$-proof
for $\varphi_n$ iff 
there exists $\Delta_i$ and $\alpha_i$ for $1 \le i \le n$ such that
\begin{enumerate}
\item $\bptop(\enGamma, \enDeltan{i}, \envarphin{i} ; \alpha_i) = \cdsucc$
      for all $1 \le i \le n$;
\item $\Delta_0 = \emptyset$, and 
      $\Delta_i \subseteq \{\varphi_1,\dots,\varphi_{i-1}\}$ 
      for all $2 \le i \le n$.
\end{enumerate}
\end{theorem}

Let $(\PPbasic, \VVbasic)$

\subsection{Phase 2: Information Sharing}

\subsection{Phase 3: Certificate Aggregation}

\end{document}